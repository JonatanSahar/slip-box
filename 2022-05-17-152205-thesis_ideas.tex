% Created 2022-05-19 Thu 09:38
% Intended LaTeX compiler: pdflatex
\documentclass[11pt]{article}
\usepackage[utf8]{inputenc}
\usepackage[T1]{fontenc}
\usepackage{graphicx}
\usepackage{longtable}
\usepackage{wrapfig}
\usepackage{rotating}
\usepackage[normalem]{ulem}
\usepackage{amsmath}
\usepackage{amssymb}
\usepackage{capt-of}
\usepackage{hyperref}
\author{Jonathan Sahar}
\date{\today}
\title{Thesis ideas}
\hypersetup{
 pdfauthor={Jonathan Sahar},
 pdftitle={Thesis ideas},
 pdfkeywords={},
 pdfsubject={},
 pdfcreator={Emacs 27.1 (Org mode 9.5)}, 
 pdflang={English}}
\usepackage{biblatex}
\addbibresource{~/google_drive/.bibliography/references.bib}
\begin{document}

\maketitle
\section{מה מעניין אותי?}
\label{sec:orgc00e8ee}
\subsubsection{איך המוח מארגן$\backslash$מקודד תנועה}
\label{sec:org4213f4f}
\subsubsection{איך זה שתנועה יכולה לערב את כל עצמנו, או לא}
\label{sec:org1d60e87}
\subsubsection{מה קורה כשתנועה נהיית יותר מחוברת - במוח? בתיאום בין השרירים?}
\label{sec:org4bd0432}
\subsubsection{מה קורה בתנועה שמתחילה ממקומות פרוקסימליים מאוד ־ משרירים גדולים}
\label{sec:org2333d0f}
\subsubsection{איך תנועה עוברת מלהיות נשלטת מאוד ברמת המיקרו, לכזאת שמתארגנת סביב פונקציה, גשטאלט}
\label{sec:org6d0aa17}
\subsubsection{איך cross education עובד? זה קשור לקונצפט של תנועה, או שזה עובר ישירות ברמה המוטורית?}
\label{sec:org1aa25f9}

\section{רעיונות למחקר: איך המוח מארגן תנועה לכדי מטרה$\backslash$פונקציה}
\label{sec:org8e15b9a}
\subsubsection{לבדוק \underline{האם אפשר למצוא ייצוג של התנועה הבסיסית בתוך פעולות שונות שמכילות אותה}.}
\label{sec:org3abb925}
\begin{itemize}
\item ‏MVPA של בלוק תנועה מסוים כנגד השתיים האחרות (ממוצע על פני הבלוק)
\item מניח שמה שמייחד את הסרטונים בבלוק היא רק התנועה הזו. האם זה נכון?
\item להשוות לסרטונים חתוכים, שלא מכילים את כל התנועה - לבלבל את סדר הפריימים ככה שהתנועה בחתיכות כמו אצל שחר - לבדוק את הקורלציה שהיא עשתה ולראות אילו איזורים מופרעים מהמקטעים הקצרים.
\item אולי אפשר לבדוק adaptation לתת התנועה (איזורים שההפעלה בהם יורדת לאורך הבלוק)
\end{itemize}
\begin{enumerate}
\item תנועות בסיסיות - כל בלוק יכיל סרטונים של פונקציות שונות שמכילות את אותה תנועה :
\label{sec:org0b2b234}
\begin{enumerate}
\item פרישת מרפק - reach אל כוס
\label{sec:org865c1d0}
\begin{enumerate}
\item לערבב
\label{sec:org289db64}
\item {\bfseries\sffamily NEXT} להקיש עליה
\label{sec:org74fa134}
\end{enumerate}
\item תפיסה - grab
\label{sec:orgd19b28a}
\begin{enumerate}
\item למזוג
\label{sec:orgb62ebb9}
\item להזיז
\label{sec:orgd96e205}
\end{enumerate}
\item להביא לכיוון הפנים
\label{sec:orgf0fe82e}
\begin{enumerate}
\item לשתות
\label{sec:orgcf5e017}
\item להסתכל
\label{sec:org615f792}
\end{enumerate}
\end{enumerate}
\item כמה כיווני מבט:
\label{sec:org5280e65}
\begin{enumerate}
\item מהצד
\label{sec:orgdbb707c}
\item מקדימה
\label{sec:org85f7f3a}
\item גוף ראשון
\label{sec:org3142bbb}
\end{enumerate}
\end{enumerate}

\subsubsection{לבדוק איך פעולות פשוטות נסכמות לכדי פעולות מורכבות יותר:}
\label{sec:orge77ae8e}
\begin{enumerate}
\item סרטונים
\label{sec:org3ed234f}
\begin{itemize}
\item לבודד את האקטיבציה של reach, grab ו turn בנפרד
\item לבודד את האקטיבציה של reach\&grab, reach\&turn ולהשוות: האם הסגנל של הפעולה המורכבת שווה$\backslash$דומה$\backslash$בקורלציה עם הסכום הלינארי (או סכום אחר) של הפעולות הפשוטות?
\item לבודד את האקטיבציה של 'grab = בלי reach (אולי עם יד ישרה) ולבדוק - באילו איזורים grab\&reach יותר דומה ל grab + reach מאשר ל 'grab
\item האם פירוק ל Independent components (ICA/PCA)z נותן את האקטיבציה של הפעולות הפשוטות?
\end{itemize}
\item פעולה עצמאית:
\label{sec:org7c34c14}
לבקש מהמתתפים לעשות פעולות נפרדות - לשלוח יד לכדור (פעולה מלאה), לאחוז בכדור כשהיד כבר קרובה אליו, ולשלוח את היד אליו בלי לאחוז (חלקי הפעולה), ולראות באיזה אופן סכום ההפעלות מתקשר להפכלה של התנועה המלאה
\end{enumerate}

\subsubsection{לבדוק איך הדימוי$\backslash$ההבנה של פעולה מתרגמים ליצירה של תכנית מוטורית}
\label{sec:orgc3b09d5}
להראות סרטונים ``רגילים'' ולעומתם סרטונים שרואים בהם רק את המצב ההתחלתי והסופי, והתנועה עצמה מוסתרת או חתוכה החוצה

שני סוגי סרטונים - כל אחד עם סוג אחר של תת-תנועה

האם נראה הבדל פעילות עבור פעולות עם תתי תנועות שונים? האם נבדיל בין תתי תנועות כמו סיבוב ימינה לעומת שמאלה
\begin{itemize}
\item אפשר גם עם RS אבל אפשר לפרש את זה כקידוד של המטרה ולא של התנועה
\end{itemize}

\subsubsection{אותה מטרה - פעולות שונות}
\label{sec:org8af0144}
לפתוח כמה סוגי קופסאות (עם מכסים שונים)
האם הקופסאות השונות לא יכולות להחש מטרות שונות?

\subsubsection{המשך לניסוי הפטיש של שחר: אותה תנועה עם כוונה אחרת:}
\label{sec:orgee2b1c4}
לעשות תנועות דומות בתוך המגנט
\begin{itemize}
\item לשלוח יד לכדור, להזיז אותו הצידה לעומת להניח אותו על הגוף
\item לסובב מפתח ימינה או שמאלה (לנעול או לפתוח)
\item הקופסא של המילטון - ממש אותה תנועה
\end{itemize}

בניסוי המקורי הן עשו decoding מהזמן שלפני הפעולה, והיה להן לחצן שאמר מתי הנבדק התחיל לזוז בפועל

\subsubsection{איך קורית האיטגרציה שך תנועות סביב פונקציה?}
\label{sec:org392831e}
לתת לנבדקים לעשות תנועה עם מטרה בשני אופנים:
\begin{itemize}
\item לחשוב רק על המטרה
\item לעשות את התנועה בלי לחשוב על המטרה הכוללת, ובלי coarticulation - לעשות תנועה תנועה
\end{itemize}

הרעיון הוא שזה יפריע לאינטגרציה שקורית באופן טבעי, וההבדל בפעילות המוחית ייצג באיזשהו אופן את האינטגרציה הזאת.
לראות את ההבדל בפעילות המוחית - MVPA?
\subsubsection{תנועה מדומיינת$\backslash$תכנון תנועה?}
\label{sec:org5bc8545}

\subsubsection{איך תנועה מדומיינת משפיעה על תנועה אמיתית שאחריה: \autocite{persichettiLayerSpecificContributionsImagined2020}}
\label{sec:org7d4231e}

\subsubsection{להשוות באותם נבדקים, תנועה אמיתית, מדומיינת ונצפית}
\label{sec:orgb98a36d}
לבדוק מה משותף לכל האקטיבציות
לעשות RS ב modalities השונות ולראות אילו איזורים מגיבים ל goal ואילו לפעולה המוטורית
\end{document}
